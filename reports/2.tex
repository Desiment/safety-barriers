\documentclass{article}
\usepackage[T2A]{fontenc}
\usepackage[russian]{babel}
\usepackage[utf8]{inputenc}

%%%%%%%%%%%%%%%%%%%%%%%%%%%% ДОП.СИМВОЛЫ  %%%%%%%%%%%%%%%%%%%%%%%%%%%%%%%%
\usepackage{amsmath}
\usepackage{amssymb}
\usepackage{latexsym}
\usepackage{amsfonts}
\usepackage{extarrows}
\usepackage{braket}
\usepackage{MnSymbol}
\usepackage{mathtools}
\usepackage{commath}

\DeclarePairedDelimiter{\ceil}{\lceil}{\rceil}
\DeclarePairedDelimiter{\floor}{\lfloor}{\rfloor}
%%%%%%%%%%%%%%%%%%%%%%%%%%%%%%%%%%%%%%%%%%%%%%%%%%%%%%%%%%%%%%%%%%%%%%%%

%%%%%%%%%%%%%%%%%%%%%%%%%%%%%  ГРАФИКА  %%%%%%%%%%%%%%%%%%%%%%%%%%%%%%%%
%Цвета:
\usepackage{color} 
\usepackage{xcolor}

%Картиночки:
\usepackage{graphicx}
\graphicspath{{pictures/}}
\DeclareGraphicsExtensions{.pdf,.png,.jpg}

%Встроенная графика 
\usepackage{tikz}
\usetikzlibrary{
    shapes.symbols,
    shapes.geometric,
    shadows,arrows.meta,
    graphs
}

\usepackage{flowchart}
%%%%%%%%%%%%%%%%%%%%%%%%%%%%%%%%%%%%%%%%%%%%%%%%%%%%%%%%%%%%%%%%%%%%%%%%

%%%%%%%%%%%%%%%%%%%%%%%%%%%%%% ВЕРСТКА 1 %%%%%%%%%%%%%%%%%%%%%%%%%%%%%%%%%
\usepackage[toc,page]{appendix}
\usepackage{hyperref}
\hypersetup{
    unicode=true,
    colorlinks=true,
    linktoc=all,  
    linkcolor=blue,
}
\usepackage{hhline}
\usepackage{subcaption}
\usepackage{float}
\usepackage{enumitem}
%%%%%%%%%%%%%%%%%%%%%%%%%%%%%%%%%%%%%%%%%%%%%%%%%%%%%%%%%%%%%%%%%%%%%%%%

%%%%%%%%%%%%%%%%%%%%%%%%%%%%%% ВЕРСТКА 2 %%%%%%%%%%%%%%%%%%%%%%%%%%%%%%%%%
% Шрифты - настройки по умолчанию.
\renewcommand{\rmdefault}{cmr}
\renewcommand{\sfdefault}{cmss}
\renewcommand{\ttdefault}{cmtt}

%Формат секции
\makeatletter
\renewcommand{\@seccntformat}[1]{}
\makeatother


%Пробел
\setlength{\parindent}{0pt}
\setlength{\parskip}{3pt}

%Размеры страницы (не забыть подогнать под принтер)
\usepackage[left=2cm,right=2cm,bottom=2cm]{geometry}

%Списки:
\setlist{topsep=1pt, itemsep=0em}
%%%%%%%%%%%%%%%%%%%%%%%%%%%%%%%%%%%%%%%%%%%%%%%%%%%%%%%%%%%%%%%%%%%%%%%%%%%%%%%%%%%%%%%%%%%%

\title{Чекпоинт II}
\author{Stardust Crusaders}
\date{21 ноября 2020 г.}
\begin{document}
\maketitle
\tableofcontents

\section*{Резюме}
Опробованы два подхода к построению кластеризации. Проведен анализ исходного датасета. Сделана выборка с ручной разметкой.

Результаты анализа:

Полученные кластеры (модель 1)

Полученные кластеры (модель 2)

\newpage
\section{Цели чекпоинта}
\begin{itemize}
    \item Развернуть окружение для работы;
    \item Подготовить и очистить данные;
    \item Первичный анализ данных;
\end{itemize}


\section{Результаты}
Исходный \texttt{.xslx} файл был преобразован в \texttt{.csv} с пятью столбцами: 
\begin{itemize}
    \item Дочернее общество:
    \item Подрядчик; 
    \item Направление деятельности;
    \item Производственный объект;
    \item Описание инцидента.
\end{itemize}

Были выделены три сущности для анализа:
\begin{itemize}
    \item Инциденты
    \begin{enumerate}[label=(\arabic*)]
        \item 
    \end{enumerate}
    \item Подрядчики
    \begin{enumerate}[label=(\arabic*)]
        \item 
    \end{enumerate}
    \item Дочерние общества
    \begin{enumerate}[label=(\arabic*)]
        \item 
    \end{enumerate}
\end{itemize}
\end{document}
